\documentclass[10pt]{article}
\begin{document}
 \paragraph{ THE IMPACT OF HUMAN ACTIVITIES ON WILDLIFE AND IT'S HABITATS }
\paragraph{by kazooba b lawrence , Stdno: 216004505 ,Regno:16/u/5830/ps}
\paragraph{INTRODUCTION}

\paragraph{Every living creature needs room to exist and reproduce. The natural home of a plant, animal or any other type of organism is known as as its habitat and maintaining this space is crucial to the ongoing survival of both individuals and species}\subparagraph{unfortunately the habitats of large numbers of earth's plants  and animal species are under threat due to impact of human beings on the planet. habitat loss is contributing to the permanent loss of species, the weakening of eco-systems and is impacting on both the overall health of the planet and quality of human life.}
\paragraph{CAUSES OF HABITAT LOSS}
\subparagraph{Human ativitivies   by far the biggest cause of habitat loss.The planet's human population  has doubled in the past 50 years and the pressure to house and feed more than seven billion people has seen incursions into previously prestine natural habitats increase dramatically at the same time. Human impact  on the earth's climate are radically changing weather patterns and as the result the spread and nature of wild habitats}
\subparagraph{The primary individual cause of loss of habitat isthe clearing of land for agriculture. An estimated 177,000 square kilometres of forests and woodland are clreared annually to make space for farming or inorder to harvest timber for fuel and wood products. Estimates suggest that the earth has lost about a half of its forest in the past  5 or so  mellenia of human activity and existence, with much of this occuring in recent decades. About 3 percent of frorest of forests have been lost since 1990s alone  }
\subparagraph{And it is not just forest clearing that leads to habitat loss. The loss of wetlands, plains, lakes and natural environments all destroy all degrade habitat as do other human activities such as introducing invasive species, polluting, trading in wildlife and engaging in wars. This destruction of habitat also involves marine zones and the ocean, with urbanisation, industrialisation and tourism all affecting habitats in coastal arears, some 40 percent of the global population live within 100 kilometres of the coastal areas placing major strains on wetlands and oceans. }
\paragraph{IMPACTS OF HUMAN ACTIVITIES }
\subparagraph{With such significant habitat destruction underway, the effects on the eco-system and wildlife are significant figures from the international union for conservation of nature (IUCN) suggest about 2000 mummals around the globe are affected by habitat loss. It is the primary threat to 85 percent of the species on the union's red list, which lists all the organisms whose existance is either vulnarable, endangered or critically endangered}
\subparagraph{The problem is particularly acute in Australia where, due to human impact more mammal species have been lost in the past 200 years than in all other continents combined. Of the 1250 plant and 390 terrestrial animal species considered threated, 964 plants and 286 animals have deforestation and resulting habitat fragmentation or degradation listed as threats. These include carnaby's cockatoo, the southern cassowary, bennet's tree kangaroo, the cape york rock wallaby, and the black flanked rock wallaby, as well as the iconic koala, recently listed as vulnarable to extinction Queensland and NSW }
\paragraph{While tree clearing is a significant cause of habitat loss in Austarlia, and other major contributing factors include altered bush fira frequency and the introduction of pest species such as cats, foxes, and weeds that make habitat unsafe for native species or outcompete them. Meanwhile on the great barrier reef, the impact of human induced climate change are altering habitats of corals, leading to large scale bleaching. over time, destruction of such habitats leads to reduced bio-diversity, weakening the earth's eco-system and ultimately posing a mojor threat to human life}
\paragraph{PROTECTING HABITATS} \subparagraph{While significant tracts of habitats have been lost and along with them many species of plants and animal steps can be taken to slow and even reverse the process. One key measure is the establishment of protected areas where human activity is restricted in order to conserve existing eco-systems and wildlife. Well palnned and well managed reserves, parks, and forests that can help to safeguard fresh water and food supplies, reduce poverty and reduce impact of natural disasters}
\subparagraph{References
\\www.google.com
}

\end{document}
